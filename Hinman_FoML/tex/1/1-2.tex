\section{Induction and recursion}

\begin{myfrm}[訳語対応]
  \begin{description}
    \item[$\cl X$-帰納法] $\cl X$-induction
    \item[$\cl X$-導出] $\cl X$-derivation
    \item[$\cl X$-閉である] $\cl X$-closed
    \item[帰納的系] inductive system
    \item[帰納的閉包] inductive closure
  \end{description}
\end{myfrm}



\begin{excfield}{注意: 定義1.2.1}
  pp. 25--26にも書かれていますが,このような$X_n$の再帰的定義は定理1.2.12によって正当化されます.

  もし$\cl X = (X, X_0, \cl H)$が帰納的系であれば,$X_0 \in \wp (X)$が成り立ちます.そこで定理1.2.12において
  \begin{eqgather}
    Z = \wp (X) \\
    z_0 = X_0 \\
    G \colon \wp (X) \times \omega \ni (x, n) \mapsto x \cup \{H (x_0, \dots, x_{k_H - 1}) \in X : H \in \cl H,\ x_0, \dots, x_{k_H - 1} \in x\} \in \wp (X)
  \end{eqgather}
  とすれば,$F (0) = X_0$かつ任意の$n \in \omega$に対して
  \[F (n + 1) = F (n) \cup \{H (x_0, \dots, x_{k_H - 1}) : H \in \cl H,\ x_0, \dots, x_{k_H - 1} \in F (n)\}\]
  であるような関数$F \colon \omega \to \wp (X)$が一意に存在することが言えます.この唯一の$F$に対する$F (n)$が,$X_n$(正確には$\cl X_n$)と書かれているのです.
\end{excfield}



\begin{excfield}{系1.2.4}
  \begin{myenum}[n]
    \item $\clos X \sbc \bigcap \{Y \sbc X : \closed{Y}{\cl X}\}$
    \item $\clos X \sprc \bigcap \{Y \sbc X : \closed{Y}{\cl X}\}$
  \end{myenum}
  を示す.

  \begin{myenum}[n]
    \item 命題1.2.3(ii)より,
    \[\{Y \sbc X : \closed{Y}{\cl X}\} \sbc \{Y \sbc X : \clos X \sbc Y\}\]
    よって,
    \[
      \clos X
      = \bigcap \{Y \sbc X : \clos X \sbc Y\}
      \sbc \bigcap\{Y \sbc X : \closed{Y}{\cl X}\}
    \]
    \item $\clos X \sbc X$および命題1.2.3(i)より,
    \[\clos X \in \{Y \sbc X : \closed{Y}{\cl X}\}\]
    よって,
    \[\clos X \sprc \bigcap \{Y \sbc X : \closed{Y}{\cl X}\}\]
  \end{myenum}
\end{excfield}



\begin{excfield}{命題1.2.7}
  $\Ra$を示す.任意の$z \in \clos X$に対して,$z$の$\cl X$-導出が存在することを$\cl X$-帰納法で示す.
  \begin{step}
    \item $z \in X_0$の場合,$(z)$は$z$の$\cl X$-導出である.
    \item $H \in \cl H$と$z_0, \dots, z_{k_H - 1} \in X$を任意に取り,$z_0, \dots, z_{k_H - 1}$の$\cl X$-導出が存在すると仮定する.それらをそれぞれ
    \begin{eqgather}
      (x_0^0, \dots, x_0^{n_0}, z_0) \\
      \vdots \\
      (x_{k_H - 1}^0, \dots, x_{k_H - 1}^{n_{k_H - 1}}, z_{k_H - 1})
    \end{eqgather}
    とすると,これらの連結に$H (z_0, \dots, z_{k_H - 1})$を追加した列
    \begin{eqalign}
      ( & x_0^0, \dots, x_0^{n_0}, z_0, \\
      & & & \dots, \\
      & & & & & x_{k_H - 1}^0, \dots, x_{k_H - 1}^{n_{k_H - 1}}, z_{k_H - 1}, \\
      & & & & & & & H (z_0, \dots, z_{k_H - 1}))
    \end{eqalign}
    は$H (z_0, \dots, z_{k_H - 1})$の$\cl X$-導出である.なぜなら,この列の$H (z_0, \dots, z_{k_H - 1})$以外の項は,仮定より定義1.2.6(i)または(ii)を満たし,また$H (z_0, \dots, z_{k_H - 1})$は定義1.2.6(ii)を満たすからである.
  \end{step}

  $\La$を示す.$(x_0, \dots, x_n)$が$x_n$の$\cl X$-導出であれば$x_n \in \clos X$であることを$\cl P (n)$と書き,これを$n$に関する帰納法で示す.
  \begin{step}
    \item $n = 0$の場合,$x_0$について定義1.2.6(ii)は成り立たないので,(i) $x_n \in X_0$が成り立つ.よって,$\cl P (n)$である.
    \item 任意の$0 \le i \le n$に対して$\cl P (i)$を仮定する.$(x_0, \dots, x_{n + 1})$は$x_{n + 1}$の$\cl X$-導出であるとすると,定義1.2.6(i)または(ii)が成り立つ.(i)の場合,$\cl P (n + 1)$である.(ii)の場合,仮定より,$x_{j_0}, \dots, x_{j_{k_H - 1}} \in \clos X$であるから,命題1.2.3(i)より,$x_{n + 1} = H (x_{j_0}, \dots, x_{j_{k_H - 1}}) \in \clos X$である.よって,$\cl P (n + 1)$である.
  \end{step}
  以上より,任意の$n \in \omega$に対して$\cl P (n)$である.よって,$(x_0, \dots, x_n)$が$z$の$\cl X$-導出であれば,$z = x_n \in \clos X$である.
\end{excfield}
