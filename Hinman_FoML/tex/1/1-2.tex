\section{Induction and recursion}

\begin{myfrm}[訳語対応]
  \begin{description}
    \item[$\cl X$-帰納法] $\cl X$-induction
    \item[$\cl X$-導出] $\cl X$-derivation
    \item[$\cl X$-閉である] $\cl X$-closed
    \item[帰納的系] inductive system
    \item[帰納的閉包] inductive closure
    \item[帰納法の仮定] induction hypothesis
  \end{description}
\end{myfrm}



\begin{excfield}{注意: 定義1.2.1}
  pp. 25--26にも書かれていますが,このような$X_n$の再帰的定義は定理1.2.12によって正当化されます.

  もし$\cl X = (X, X_0, \cl H)$が帰納的系であれば,$X_0 \in \wp (X)$が成り立ちます.そこで定理1.2.12において
  \begin{eqgather}
    Z = \wp (X) \\
    z_0 = X_0 \\
    G \colon \wp (X) \times \omega \ni (x, n) \mapsto x \cup \{H (x_0, \dots, x_{k_H - 1}) \in X : H \in \cl H,\ x_0, \dots, x_{k_H - 1} \in x\} \in \wp (X)
  \end{eqgather}
  とすれば,$F (0) = X_0$かつ任意の$n \in \omega$に対して
  \[F (n + 1) = F (n) \cup \{H (x_0, \dots, x_{k_H - 1}) : H \in \cl H,\ x_0, \dots, x_{k_H - 1} \in F (n)\}\]
  であるような関数$F \colon \omega \to \wp (X)$が一意に存在することが言えます.この唯一の$F$に対する$F (n)$が,$X_n$(正確には$\cl X_n$)と書かれているのです.
\end{excfield}



\begin{excfield}{系1.2.4}
  \begin{myenum}[n]
    \item $\clos X \sbc \bigcap \{Y \sbc X : \closed{Y}{\cl X}\}$
    \item $\clos X \sprc \bigcap \{Y \sbc X : \closed{Y}{\cl X}\}$
  \end{myenum}
  を示す.

  \begin{myenum}[n]
    \item 命題1.2.3(ii)より,
    \[\{Y \sbc X : \closed{Y}{\cl X}\} \sbc \{Y \sbc X : \clos X \sbc Y\}\]
    よって,
    \[
      \clos X
      = \bigcap \{Y \sbc X : \clos X \sbc Y\}
      \sbc \bigcap\{Y \sbc X : \closed{Y}{\cl X}\}
    \]
    \item $\clos X \sbc X$および命題1.2.3(i)より,
    \[\clos X \in \{Y \sbc X : \closed{Y}{\cl X}\}\]
    よって,
    \[\clos X \sprc \bigcap \{Y \sbc X : \closed{Y}{\cl X}\}\]
  \end{myenum}
\end{excfield}



\begin{excfield}{注意: 定理1.2.5}
  特に,$X = \clos X$とすれば,$(\clos X, X_0, \cl H)$は帰納的系なので,帰納法の仮定$\cl P (x_0), \dots, \cl P (x_{k_H - 1})$は$x_0, \dots, x_{k_H - 1} \in \rx{\clos X}$のときのみを考えればよいことが分かります.
\end{excfield}



\begin{excfield}{定理1.2.12}
  証明をもう少し詳しく書きます.\\
  \dotfill

  \begin{step}
    \item $z_n = z$,かつ任意の$i < n$に対して$z_{i + 1} = G (z_i, i)$であるような$z_1, \dots, z_n$が存在する,ということを$\cl P (z, n)$と書き,また$\cl P (z, n)$なる$z$が一意に存在することを$\cl Q (n)$と書く.任意の$n \ge 1$に対して$\cl Q (n)$を帰納法で示す.
    \begin{step}
      \item $z = z_0$であり,また任意の$i < 0$に対して自明に$z_{i + 1} = G (z_i, i)$であるから,$\cl P (z, 0)$である.このような$z$は$z_0$のみであるから,$\cl Q (0)$である.
      \item $\cl Q (n)$を仮定し,$\cl Q (n + 1)$を示す.仮定より,$\cl P (z, n)$なる$z$が一意に存在するので,それを$\overline{z}$と書く.$\cl P (\overline{z}, n)$であるから,適当な$z_1, \dots, z_n$を取れば
      \begin{eqalign}
        z_1 & {}= G (z_0, 0) \\
        & \vdots \\
        z_n & {}= G (z_{n - 1}, n - 1) \\
        \overline{z} & {}= z_n
      \end{eqalign}
      である.ここで$z = z_{n + 1} = G (\overline{z}, n)$とすれば$\cl P (z, n + 1)$であり,そのような$z$は$G (\overline{z}, n)$のみであるから,$\cl Q (n + 1)$である.
    \end{step}
  \end{step}
  $\cl P (z, n)$を満たす,この一意に存在する$z$を$z^n$と書く.
  \begin{step}[resume]
    \item 関数$F$を$F \colon \omega \ni n \mapsto z^n \in Z$によって定める.
    \begin{step}
      \item $F (0) = z^0 = z_0$である.
      \item $n \in \omega$に対して,$z^n$の定義より,$F (n + 1) = z^{n + 1} = G (z_n, n)$である.
    \end{step}
  \end{step}
\end{excfield}



\begin{excfield}{演習1.2.18}
  $\clos X \sbc X$は,$\clos X$の定義から従う.

  以下,
  \begin{myenum*}
    \item $X_0 \sbc X$
    \item 任意の$H \in \cl H$と$x_0, \dots, x_{k_H - 1} \in \clos X$に対して,$H (x_0, \dots, x_{k_H - 1}) \in \clos X$
  \end{myenum*}
  を示す.
  \begin{myenum}
    \item $Y$が$\clos X$-閉であれば$X_0 \sbc Y$であるから,
    \[
      \clos X
      = \bigcap \{Y \sbc X : \closed{Y}{\cl X}\}
      \sprc \{Y \sbc X : X_0 \sbc Y\}
      = X_0
    \]
    より,$X_0 \sbc \clos X$を得る.
    \item $H \in \cl H$,$x_0, \dots, x_{k_H - 1} \in \clos X$とする.$\cl X$-閉である$Y$を任意に取る.
    \[
      \clos X
      = \bigcap \{Y \sbc X : \closed{Y}{\cl X}\}
      = \bigcap \{Y : \closed{Y}{\cl X}\}
    \]
    より,$x_0, \dots, x_{k_H - 1} \in Y$であり,$Y$は$\cl X$-閉であるから,$H (x_0, \dots, x_{k_H - 1}) \in Y$である.よって,
    \[
      H (x_0, \dots, x_{k_H - 1})
      \in \bigcap \{Y \sbc X : \closed{Y}{\cl X}\}
      = \clos X
    \]
    である.
  \end{myenum}
\end{excfield}



\begin{excfield}{演習1.2.19}
  一意可読性より,任意の$x \in X_{n + 1} \sim X_n$に対して,$x = H (x_0, \dots, x_{k_H - 1})$なる$H \in \cl H$と$x_0, \dots, x_{k_H - 1} \in X_n$がそれぞれ一意に存在する.したがって,任意の$n \in \omega$に対し,関数$F^u_n \colon X_n \to Z$を以下のように再帰的に定義できる.
  \begin{eqalign}
    F^u_0 (x) & {}= F_0 (x, u) & & \text{if $x \in X_0$} \\
    F^u_{n + 1} (x) & {}= F^u_n (x) & & \text{if $x \in X_n$} \\
    F^u_{n + 1} (H (x_0, \dots, x_{k_H - 1})) & {}= G_H (F^u_n (x_0), \dots, F^u_n (x_{k_H - 1}), x_0, \dots, x_{k_H - 1}, u) & & \text{if $x \in X_{n + 1} \sim X_n$}
  \end{eqalign}
  $i_x$を$x \in X_n$なる最小の$n \in \omega$とし,関数$F \colon \clos X \times U \to Z$を$F (x, u) = F^u_{i_x} (x)$によって定義すると,$x \in X_n$なる任意の$n$,つまり$n \ge i_x$に対して,$F (x) = F^u_{i_x} (x) = \dots = F^u_n (x)$である.よって,
  \begin{step}
    \item 任意の$x \in X_0$に対して$F (x, u) = F^u_0 (x) = F_0 (x, u)$であるから,$F$は$F_0$の拡張である.
    \item 任意の$H \in \cl H$,$u \in U$,$x_0, \dots, x_{k_H - 1} \in \clos X$に対して,適当な$n \in \omega$が存在して,
    \begin{eqalign}
      F (H (x_0, \dots, x_{k_H - 1}), u)
      & {}= F^u_{n + 1} (H (x_0, \dots, x_{k_H - 1})) \\
      & {}= G_H (F^u_n (x_0), \dots, F^u_n (x_{k_H - 1}), x_0, \dots, x_{k_H - 1}, u) \\
      & {}= G_H (F (x_0, u), \dots, F (x_{k_H - 1}, u), x_0, \dots, x_{k_H - 1}, u)
    \end{eqalign}
    である.よって,
    \[
      F (H (x_0, \dots, x_{k_H - 1}), u)
      = G_H (F (x_0, u), \dots, F (x_{k_H - 1}, u), x_0, \dots, x_{k_H - 1}, u)
    \]
    である.
  \end{step}

  $F$の一意性を示す.いま,関数$F' \colon \clos X  \times U \to Z$も$F_0$の拡張であり,かつ
  \[
    F' (H (x_0, \dots, x_{k_H - 1}), u)
    = G_H (F' (x_0, u), \dots, F' (x_{k_H - 1}, u), x_0, \dots, x_{k_H - 1}, u)
  \]
  を満たすとする.任意の$x \in \clos X$と$u \in U$に対して$F (x, u) = F' (x, u)$が成り立ち,したがって$F = F'$が成り立つことを,$\clos X$-帰納法で示す.
  \begin{step}
    \item $x \in X_0$の場合,$F (x, u) = F^u_0 (x) = F_0 (x, u) = F'^u_0 (x) = F' (x, u)$である.
    \item $H \in \cl H$,$x_0, \dots, x_{k_H - 1} \in \clos X$を任意に取り,$F (x_i, u) = F' (x_i, u)$を仮定する($0 \le i \le k_H - 1$)と,
    \begin{eqalign}
      F (H (x_0, \dots, x_{k_H - 1}), u)
      & {}= G_H (F (x_0, u), \dots, F (x_{k_H - 1}, u), x_0, \dots, x_{k_H - 1}, u) \\
      & {}= G_H (F' (x_0, u), \dots, F' (x_{k_H - 1}, u), x_0, \dots, x_{k_H - 1}, u) \\
      & {}= F' (H (x_0, \dots, x_{k_H - 1}), u)
    \end{eqalign}
  \end{step}
\end{excfield}



\begin{excfield}{演習1.2.21}
  $\Ra$を示す.任意の$z \in \clos X$に対して,$z$の$\cl X$-導出が存在することを$\cl X$-帰納法で示す.
  \begin{step}
    \item $z \in X_0$の場合,$(z)$は$z$の$\cl X$-導出である.
    \item $H \in \cl H$と$z_0, \dots, z_{k_H - 1} \in X$を任意に取り,$z_0, \dots, z_{k_H - 1}$の$\cl X$-導出が存在すると仮定する.それらをそれぞれ
    \begin{eqgather}
      (x_0^0, \dots, x_0^{n_0}, z_0) \\
      \vdots \\
      (x_{k_H - 1}^0, \dots, x_{k_H - 1}^{n_{k_H - 1}}, z_{k_H - 1})
    \end{eqgather}
    とすると,これらの連結に$H (z_0, \dots, z_{k_H - 1})$を追加した列
    \begin{eqalign}
      & (x_0^0, \dots, x_0^{n_0}, z_0, \\
      & & & \dots, \\
      & & & & & x_{k_H - 1}^0, \dots, x_{k_H - 1}^{n_{k_H - 1}}, z_{k_H - 1}, \\
      & & & & & & & H (z_0, \dots, z_{k_H - 1}))
    \end{eqalign}
    は$H (z_0, \dots, z_{k_H - 1})$の$\cl X$-導出である.なぜなら,この列の$H (z_0, \dots, z_{k_H - 1})$以外の項は,仮定より定義1.2.6(i)または(ii)を満たし,また$H (z_0, \dots, z_{k_H - 1})$は定義1.2.6(ii)を満たすからである.
  \end{step}

  $\La$を示す.$(x_0, \dots, x_n)$が$x_n$の$\cl X$-導出であれば$x_n \in \clos X$である,ということを$\cl P (n)$と書き,これを$n$に関する帰納法で示す.
  \begin{step}
    \item $n = 0$の場合,$x_0$について定義1.2.6(ii)は成り立たないので,(i) $x_0 \in X_0$が成り立つ.これと$X_0 \sbc \clos X$より,$\cl P (0)$である.
    \item 任意の$0 \le i \le n$に対して$\cl P (i)$を仮定する.$(x_0, \dots, x_{n + 1})$が$x_{n + 1}$の$\cl X$-導出であるとすると,定義1.2.6(i)または(ii)が成り立つ.(i)の場合,$X_0 \sbc \clos X$より$\cl P (n + 1)$である.(ii)の場合,仮定より,$x_{j_0}, \dots, x_{j_{k_H - 1}} \in \clos X$であるから,命題1.2.3(i)より,$x_{n + 1} = H (x_{j_0}, \dots, x_{j_{k_H - 1}}) \in \clos X$である.よって,$\cl P (n + 1)$である.
  \end{step}
  以上より,任意の$n \in \omega$に対して$\cl P (n)$である.よって,$(x_0, \dots, x_n)$が$z$の$\cl X$-導出であれば,$z = x_n \in \clos X$である.
\end{excfield}
