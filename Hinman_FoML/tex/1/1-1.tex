\begin{excfield}{注意: 命題1.1.5の補題(4)の証明}
  補題(4)の証明は,$\phi_0 \dots \phi_k$の長さに関する帰納法に基づいていますが,帰納法のbasisである,長さが$1$の場合に(4)が正しいことの証明が省略されています.これは次のように証明できます.$\phi_0 \dots \phi_k$と$\psi_0 \dots \psi_l$の長さに関して$1 > k, l$であるため,$k = l = 0$でしかありえず,したがって$\phi_0 = \psi_0$となります.
\end{excfield}



\begin{excfield}{演習1.1.10}
  以下のように定義する.
  \begin{dfn}[中置記法での$L$--命題論理式の集合]
    \begin{thmlist}[n]
      \item $\Sent_0 \dfeq \text{$L$--原子命題論理式の集合}$
      \item 任意の$n \in \omega$に対して,
      \begin{eqalign}
        \Sent_{n + 1} \dfeq{} & \Sent_n \cup \{(\lnot \phi) : \phi \in \Sent_n\} \\
        & {}\cup \{(\phi \bullet \psi) : \phi, \psi \in \Sent_n, \text{$\bullet$は$\lor, \land, \ra, \lra$のいずれか}\}
      \end{eqalign}
      \item $\Sent_L \dfeq \bigcup_{n \in \omega} \Sent_n$
    \end{thmlist}
  \end{dfn}

  次は補題1.1.3および命題1.1.4と全く同じ方法で証明できる.
  \begin{prp}[$L$--命題論理式の帰納法による証明][i_1_10_ind][$L$--命題論理式の帰納法]
    $L$--表現に関する任意の性質$\cl P$に対して,
    \begin{myenum}
      \item 任意の$L$--原子命題論理式について$\cl P$が成り立ち,かつ
      \item 任意の$L$--命題論理式$\phi, \psi$に対し,$\phi, \psi$について$\cl P$が成り立つならば,$(\lnot \phi)$,$(\phi \lor \psi)$,$(\phi \land \psi)$,$(\phi \ra \psi)$,$(\phi \lra \psi)$についても$\cl P$が成り立つ
    \end{myenum}
    ならば,任意の$L$--命題論理式に対して$\cl P$が成り立つ.
  \end{prp}

  次を証明する.
  \begin{prp}[一意可読性][i_1_10_ur]
    任意の$L$--命題論理式$\theta$に対して,以下のちょうど1つが成り立つ.
    \begin{myenum}
      \item \label{i_1_10_ur_1}
      $\theta$は$L$--原子命題論理式である.
      \item \label{i_1_10_ur_2}
      $\theta = (\lnot \phi)$なる$L$--命題論理式$\phi$が存在する.
      \item \label{i_1_10_ur_3}
      $\theta = (\phi \lor \psi)$なる$L$--命題論理式$\phi, \psi$がそれぞれ一意に存在する.
      \item \label{i_1_10_ur_4}
      $\theta = (\phi \land \psi)$なる$L$--命題論理式$\phi, \psi$がそれぞれ一意に存在する.
      \item \label{i_1_10_ur_5}
      $\theta = (\phi \ra \psi)$なる$L$--命題論理式$\phi, \psi$がそれぞれ一意に存在する.
      \item \label{i_1_10_ur_6}
      $\theta = (\phi \lra \psi)$なる$L$--命題論理式$\phi, \psi$がそれぞれ一意に存在する.
    \end{myenum}
  \end{prp}

  そのために次を証明する.以下,$\bullet$は$\lor, \land, \ra, \lra$のいずれかとする.
  \begin{lem}
    \begin{thmlist}[n]
      \item \label{i_1_10_lem_preq}
      $L$--命題論理式に含まれる$($の個数と$)$の個数は同じである.
      \item \label{i_1_10_lem_prg}
      $L$--命題論理式の真の始切片\footnote{演習1.1.11参照.}に含まれる$($の個数は$)$の個数より多い.
      \item \label{i_1_10_lem_iniseg}
      $L$--命題論理式の真の始切片は$L$--命題論理式ではない.
      \item \label{i_1_10_lem_conn}
      $\bullet'$は$\lor, \land , \ra, \lra$のいずれかとし,$\phi, \psi, \phi', \psi'$は$L$--命題論理式とする.$(\phi \bullet \psi) = (\phi' \mathbin{\bullet'} \psi')$ならば,$\phi = \phi'$,$\bullet = \bullet'$,$\psi = \psi'$である.
    \end{thmlist}

    \tcblower

    \begin{myenum}[n]
      \item $\phi$に対してこれが成り立つことを$\cl P (\phi)$と書く.任意の$L$--命題論理式$\phi$に対して$\cl P (\phi)$を\nameref{i_1_10_ind}で示す.
      \begin{step}
        \item $\phi$が$L$--原始命題論理式の場合,$($と$)$は含まれないので,$\cl P (\phi)$である.
        \item $L$--命題論理式$\phi, \psi$を任意に取り,$\cl P (\phi)$と$\cl P (\psi)$を仮定する.仮定より,$\cl P ((\lnot \phi))$,$\cl P (\phi \bullet \psi)$であることは明らかである.
      \end{step}
      \item $\phi$に対してこれが成り立つことを$\cl P (\phi)$と書く.任意の$L$--命題論理式$\phi$に対して$\cl P (\phi)$を\nameref{i_1_10_ind}で示す.
      \begin{step}
        \item $\phi$が$L$--原始命題論理式の場合,真の始切片が存在しないので,$\cl P (\phi)$である.
        \item $L$--命題論理式$\phi, \psi$を任意に取り,$\cl P (\phi)$と$\cl P (\psi)$を仮定する.$(\lnot \phi)$,$(\phi \bullet \psi)$のいずれについても,その真の始切片は右端の$\noprint{(})$を持たず,従って(i)より,そこに含まれる$($の個数は$)$の個数より多い.つまり,$\cl P ((\lnot \phi))$,$\cl P ((\phi \bullet \psi))$である.
      \end{step}
      \item \itemref{i_1_10_lem_preq}と\itemref{i_1_10_lem_prg}から従う.
      \item $(\phi \bullet \psi) = (\phi' \mathbin{\bullet'} \psi')$であれば,$\noprint{(}\phi \bullet \psi) = \noprint{(}\phi' \mathbin{\bullet'} \psi')$であり,\itemref{i_1_10_lem_iniseg}より,$\phi$と$\phi'$の一方は他方の真の始切片になりえないので,$\phi = \phi'$である.よって,$\bullet = \bullet'$,次いで$\psi = \psi'$が従う.
    \end{myenum}
  \end{lem}

  \cref{i_1_10_ur}を証明する.

  (i)--(vi)のちょうど1つが$\theta$について成り立つことを$\cl P (\theta)$と書く.任意の$L$--命題論理式$\theta$に対して$\cl P (\theta)$を\nameref{i_1_10_ind}で示す.
  \begin{step}
    \item \label{i_1_10_prf_atm}
    $\theta$が$L$--原始命題論理式の場合,(i)のみが成り立つので,$\cl P (\theta)$である.
  \end{step}
  $L$--命題論理式$\theta, \theta'$を任意に取り,$\cl P (\theta)$と$\cl P (\theta')$を仮定する.
  \begin{step}[resume]
    \item \label{i_1_10_prf_lnot}
    $(\lnot \theta) = (\lnot \phi)$なる$L$--命題論理式$\phi$は一意に存在するので,(ii)が成り立ち,また左端から2番目の記号が$\lnot$であるのは(ii)の場合だけである.よって$\cl P ((\lnot \theta))$である.
    \item \label{i_1_10_prf_lor}
    $(\theta \lor \theta') = (\phi \lor \psi)$なる$L$--命題論理式$\phi, \psi$の存在は明らかである($\theta, \theta'$自身).また,$(\theta \lor \theta')$について,\ref{i_1_10_prf_atm},\ref{i_1_10_prf_lnot}と同様の理由で,(i)と(ii)は成り立たない.また\cref{i_1_10_lem_conn}より,(iii)--(vi)のうち(iii)のみが成り立つ.よって$\cl P ((\theta \lor \theta'))$である.
    \item \ref{i_1_10_prf_lor}と同様に,$\cl P ((\theta \land \theta'))$,$\cl P ((\theta \ra \theta'))$,$\cl P ((\theta \lra \theta'))$である.
  \end{step}
\end{excfield}
