\section{The propositional language}

\begin{myfrm}[訳語対応]
  \begin{description}
    \item[一意可読性] unique readability
    \item[原子文] atomic sentence
    \item[真の始切片] proper initial segment
    \item[命題記号] sentence symbol
    \item[文] sentence
    \item[文の帰納法] sentence induction
  \end{description}
\end{myfrm}



\begin{excfield}{注意: 命題1.1.5の補題(4)の証明}
  補題(4)の証明は,$\phi_0 \dots \phi_k$の長さに関する帰納法に基づいていますが,帰納法の基底である,長さが$1$の場合に(4)が正しいことの証明が省略されています.これは次のように証明できます.$\phi_0 \dots \phi_k$と$\psi_0 \dots \psi_l$の長さに関して$1 > k, l$であるため,$k = l = 0$でしかありえず,したがって$\phi_0 = \psi_0$となります.
\end{excfield}



\begin{excfield}{演習1.1.10}
  以下のように定義する.
  \begin{dfn}[中置記法での$L$--文の集合]
    \begin{thmlist}[n]
      \item $\Sent_0 \dfeq \text{$L$--原子文の集合}$
      \item 任意の$n \in \omega$に対して,
      \begin{eqalign}
        \Sent_{n + 1} \dfeq{} & \Sent_n \cup \{(\lnot \phi) : \phi \in \Sent_n\} \\
        & {}\cup \{(\phi \bullet \psi) : \phi, \psi \in \Sent_n, \text{$\bullet$は$\lor, \land, \ra, \lra$のいずれか}\}
      \end{eqalign}
      \item $\Sent_L \dfeq \bigcup_{n \in \omega} \Sent_n$
    \end{thmlist}
  \end{dfn}

  次は補題1.1.3および命題1.1.4と全く同じ方法で証明できる.
  \begin{prp}[$L$--文の帰納法による証明][i_1_10_ind][$L$--文の帰納法]
    $L$--表現に関する任意の性質$\cl P$に対して,
    \begin{myenum}
      \item 任意の$L$--原子文について$\cl P$が成り立ち,かつ
      \item 任意の$L$--文$\phi, \psi$に対し,$\phi, \psi$について$\cl P$が成り立つならば,$(\lnot \phi)$,$(\phi \lor \psi)$,$(\phi \land \psi)$,$(\phi \ra \psi)$,$(\phi \lra \psi)$についても$\cl P$が成り立つ
    \end{myenum}
    ならば,任意の$L$--文に対して$\cl P$が成り立つ.
  \end{prp}

  次を証明する.
  \begin{prp}[一意可読性][i_1_10_ur]
    任意の$L$--文$\theta$に対して,以下のちょうど1つが成り立つ.
    \begin{myenum}
      \item \label{i_1_10_ur_1}
      $\theta$は$L$--原子文である.
      \item \label{i_1_10_ur_2}
      $\theta = (\lnot \phi)$なる$L$--文$\phi$が存在する.
      \item \label{i_1_10_ur_3}
      $\theta = (\phi \lor \psi)$なる$L$--文$\phi, \psi$がそれぞれ一意に存在する.
      \item \label{i_1_10_ur_4}
      $\theta = (\phi \land \psi)$なる$L$--文$\phi, \psi$がそれぞれ一意に存在する.
      \item \label{i_1_10_ur_5}
      $\theta = (\phi \ra \psi)$なる$L$--文$\phi, \psi$がそれぞれ一意に存在する.
      \item \label{i_1_10_ur_6}
      $\theta = (\phi \lra \psi)$なる$L$--文$\phi, \psi$がそれぞれ一意に存在する.
    \end{myenum}
  \end{prp}

  そのために次を証明する.以下,$\bullet$は$\lor, \land, \ra, \lra$のいずれかとする.
  \begin{lem}
    \begin{thmlist}[n]
      \item \label{i_1_10_lem_preq}
      $L$--文に含まれる$($の個数と$)$の個数は同じである.
      \item \label{i_1_10_lem_prg}
      $L$--文の真の始切片\footnote{演習1.1.11参照.}に含まれる$($の個数は$)$の個数より多い.
      \item \label{i_1_10_lem_iniseg}
      $L$--文の真の始切片は$L$--文ではない.
      \item \label{i_1_10_lem_conn}
      $\bullet'$は$\lor, \land , \ra, \lra$のいずれかとし,$\phi, \psi, \phi', \psi'$は$L$--文とする.$(\phi \bullet \psi) = (\phi' \mathbin{\bullet'} \psi')$ならば,$\phi = \phi'$,$\bullet = \bullet'$,$\psi = \psi'$である.
    \end{thmlist}

    \tcblower

    \begin{myenum}[n]
      \item $\phi$に対してこれが成り立つことを$\cl P (\phi)$と書く.任意の$L$--文$\phi$に対して$\cl P (\phi)$を\nameref{i_1_10_ind}で示す.
      \begin{step}
        \item $\phi$が$L$--原始文の場合,$($と$)$は含まれないので,$\cl P (\phi)$である.
        \item $L$--文$\phi, \psi$を任意に取り,$\cl P (\phi)$と$\cl P (\psi)$を仮定する.仮定より,$\cl P ((\lnot \phi))$,$\cl P (\phi \bullet \psi)$であることは明らかである.
      \end{step}
      \item $\phi$に対してこれが成り立つことを$\cl P (\phi)$と書く.任意の$L$--文$\phi$に対して$\cl P (\phi)$を\nameref{i_1_10_ind}で示す.
      \begin{step}
        \item $\phi$が$L$--原始文の場合,真の始切片が存在しないので,$\cl P (\phi)$である.
        \item $L$--文$\phi, \psi$を任意に取り,$\cl P (\phi)$と$\cl P (\psi)$を仮定する.$(\lnot \phi)$,$(\phi \bullet \psi)$のいずれについても,その真の始切片は右端の$\noprint{(})$を持たず,従って(i)より,そこに含まれる$($の個数は$)$の個数より多い.つまり,$\cl P ((\lnot \phi))$,$\cl P ((\phi \bullet \psi))$である.
      \end{step}
      \item \itemref{i_1_10_lem_preq}と\itemref{i_1_10_lem_prg}から従う.
      \item $(\phi \bullet \psi) = (\phi' \mathbin{\bullet'} \psi')$であれば,$\noprint{(}\phi \bullet \psi) = \noprint{(}\phi' \mathbin{\bullet'} \psi')$であり,\itemref{i_1_10_lem_iniseg}より,$\phi$と$\phi'$の一方は他方の真の始切片になりえないので,$\phi = \phi'$である.よって,$\bullet = \bullet'$,次いで$\psi = \psi'$が従う.
    \end{myenum}
  \end{lem}

  \cref{i_1_10_ur}を証明する.

  (i)--(vi)のちょうど1つが$\theta$について成り立つことを$\cl P (\theta)$と書く.任意の$L$--文$\theta$に対して$\cl P (\theta)$を\nameref{i_1_10_ind}で示す.
  \begin{step}
    \item \label{i_1_10_prf_atm}
    $\theta$が$L$--原始文の場合,(i)のみが成り立つので,$\cl P (\theta)$である.
  \end{step}
  $L$--文$\theta, \theta'$を任意に取り,$\cl P (\theta)$と$\cl P (\theta')$を仮定する.
  \begin{step}[resume]
    \item \label{i_1_10_prf_lnot}
    $(\lnot \theta) = (\lnot \phi)$なる$L$--文$\phi$は一意に存在するので,(ii)が成り立ち,また左端から2番目の記号が$\lnot$であるのは(ii)の場合だけである.よって$\cl P ((\lnot \theta))$である.
    \item \label{i_1_10_prf_lor}
    $(\theta \lor \theta') = (\phi \lor \psi)$なる$L$--文$\phi, \psi$の存在は明らかである($\theta, \theta'$自身).また,$(\theta \lor \theta')$について,\ref{i_1_10_prf_atm},\ref{i_1_10_prf_lnot}と同様の理由で,(i)と(ii)は成り立たない.また\cref{i_1_10_lem_conn}より,(iii)--(vi)のうち(iii)のみが成り立つ.よって$\cl P ((\theta \lor \theta'))$である.
    \item \ref{i_1_10_prf_lor}と同様に,$\cl P ((\theta \land \theta'))$,$\cl P ((\theta \ra \theta'))$,$\cl P ((\theta \lra \theta'))$である.
  \end{step}
\end{excfield}



\begin{excfield}{演習1.1.11}
  $n \in \omega$に対し,長さ$n$の任意の$L$--文$\phi$の真の始切片は$L$--文ではない,ということを$\cl P (n)$と書く.任意の$n$に対して$\cl P (n)$を帰納法で示す.
  \begin{step}
    \item $n = 1$の場合,真の始切片が存在しないので,$\cl P (1)$である.
  \end{step}
  任意の$1 \le i \le n$に対して$\cl P (i)$を仮定し,長さ$n + 1$の$L$--文$\phi$を任意に取る(もしそのような$\phi$が存在しなければ,自明に$\cl P (n + 1)$である).命題1.1.5の証明の(1)と(2)より,$\phi$について(i)--(vi)のちょうど1つが成り立つ.
  \begin{step}[resume]
    \item $\phi$が$L$--原始文の場合,真の始切片が存在しないので,$\cl P (n + 1)$である.
    \item $\phi = \lnot \psi$の場合,その真の始切片は$\lnot$か$\lnot S$の形である($S$は$\psi$の真の始切片).前者は$L$--文ではない.後者は,帰納法の仮定より$S$は$L$--文ではないので,命題1.1.5(ii)が成り立たず,また左端の記号が異なるので,(ii)以外も成り立たない.よって,$\phi$の真の始切片は$L$--文ではないので,$\cl P (n + 1)$である.
    \item \label{i_1_11_lor}
    $\phi = \lorord \psi \psi'$の場合,その真の始切片は以下のいずれかの形であり,仮にそれが$L$--文であれば,命題1.1.5(iii)が成り立つはずである.
    \begin{step}
      \item $\lor$.これは$L$--文ではない.
      \item $\lorord S$($S$は$\psi$の真の始切片).帰納法の仮定より,$S$は$L$--文ではないので,命題1.1.5(iii)は成り立たない.よって,これは$L$--文ではない.
      \item $\lorord \psi$.帰納法の仮定より,$\psi$の真の始切片$\chi$は$L$--文ではない.したがって,$\lorord \psi = \lorord \chi \chi'$なる$L$--文$\chi, \chi'$は存在しないので,命題1.1.5(iii)は成り立たない.よって,これは$L$--文ではない.
      \item $\lorord \psi S$($S$は$\psi'$の真の始切片).$L$--文$\chi$の長さが$n$未満であれば,帰納法の仮定より,$\psi$と$\chi$の一方が他方の真の始切片になることはない.したがって,$\lorord \psi S = \lorord \chi \chi'$なる$L$--文$\chi, \chi'$は存在しないので,命題1.1.5(iii)は成り立たない.よって,これは$L$--文ではない.
    \end{step}
    以上より,$\cl P (n + 1)$である.
    \item $\phi = \bulletord \psi \psi'$($\bullet = \landord, \raord, \lraord$)の場合も,\ref{i_1_11_lor}と同様にして$\cl P (n + 1)$を証明できる.
  \end{step}
  証明は以上である.

  この結果が補題(4)の代わりになることは次のようにして分かる.$\bulletord \phi \psi = \bulletord \phi' \psi'$であるとする.$\phi$と$\phi'$の一方が他方の真の始切片になることはないので,$\phi = \phi'$であり,したがって$\psi = \psi'$である.
\end{excfield}



\begin{excfield}{演習1.1.12}
  任意の$\phi \in \Sent_{n + 1} \sim \Sent_n$に対し,定義1.1.2と一意可読性より,以下のいずれかちょうど1つが成り立つ.
  \begin{myenum}
    \item $\phi = \lnot \psi$なる$\psi \in \Sent_n$が一意に存在する.
    \item $\phi = \lorord \psi \psi'$なる$\psi, \psi' \in \Sent_n$がそれぞれ一意に存在する.
    \item $\phi = \landord \psi \psi'$なる$\psi, \psi' \in \Sent_n$がそれぞれ一意に存在する.
    \item $\phi = \raord \psi \psi'$なる$\psi, \psi' \in \Sent_n$がそれぞれ一意に存在する.
    \item $\phi = \lraord \psi \psi'$なる$\psi, \psi' \in \Sent_n$がそれぞれ一意に存在する.
  \end{myenum}

  したがって,任意の$n \in \omega$に対し,関数$F_{n + 1} \colon \Sent_n \to Z$を以下のように再帰的に定義できる.
  \begin{eqalign}
    F_{n + 1} (\phi) & {}= F_n (\phi) & & \text{if $\phi \in \Sent_n$} \\
    F_{n + 1} (\lnot \phi) & {}= G_\lnot (F_n (\phi)) & & \text{if $\lnot \phi \in \Sent_{n + 1} \sim \Sent_n$} \\
    F_{n + 1} (\bulletord \phi \psi) & {}= G_\bullet (F_n (\phi), F_n(\psi)) & & \text{if $\bulletord \phi \psi \in \Sent_{n + 1} \sim \Sent_n$}
  \end{eqalign}
  $i_\phi$を$\phi \in \Sent_n$なる最小の$n \in \omega$とし,関数$F \colon \Sent_L \to Z$を$F (\phi) = F_{i_\phi} (\phi)$によって定義すると,$\phi \in \Sent_n$なる任意の$n$,つまり$n \ge i_\phi$に対して,$F (\phi) = F_{i_\phi} (\phi) = F_{i_\phi + 1} (\phi) = \dots = F_n (\phi)$である.よって,
  \begin{step}
    \item 任意の$\phi \in \Sent_0$に対して$F (\phi) = F_0 (\phi)$であるから,$F$は$F_0$の拡張である.
    \item 任意の$\phi \in \Sent_L$に対して,適当な$n \in \omega$が存在して,
    \begin{eqgather}
      F (\lnot \phi) = F_{n + 1} (\lnot \phi) = G_\lnot (F_n (\phi)) = G_\lnot (F (\phi)) \\
      F (\bulletord \phi \psi) = F_{n + 1} (\bulletord \phi \psi) = G_\bullet (F_n (\phi), F_n(\psi)) = G_\bullet (F (\phi), F(\psi))
    \end{eqgather}
    となる.よって,
    \begin{eqalign}
      F (\lnot \phi) & {}= G_\lnot (F (\phi)) \\
      F (\bulletord \phi \psi) & {}= G_\bullet (F (\phi), F (\psi))
    \end{eqalign}
    である.
  \end{step}

  $F$の一意性を示す.いま,関数$F' \colon \Sent_L \to Z$も$F_0$の拡張であり,かつ
  \begin{eqalign}
    F' (\lnot \phi) & {}= G_\lnot (F' (\phi)) \\
    F' (\bulletord \phi \psi) & {}= G_\bullet (F' (\phi), F' (\psi))
  \end{eqalign}
  を満たすとする.任意の$\phi \in \Sent_L$に対して$F (\phi) = F' (\phi)$が成り立ち,したがって$F = F'$が成り立つことを,$L$--文の帰納法で示す.
  \begin{step}
    \item $\phi \in \Sent_0$の場合,$F (\phi) = F_0 (\phi) = F' (\phi)$である.
  \end{step}
  $\phi , \psi \in \Sent_L$を任意に取り,$F (\phi) = F' (\phi)$,$F (\psi) = F' (\psi)$を仮定する.
  \begin{step}[resume]
    \item 仮定より,
    \[F (\lnot \phi) = G_\lnot (F (\phi)) = G_\lnot (F' (\phi)) = F' (\lnot \phi)\]
    \item 仮定より,
    \[F (\bulletord \phi \psi) = G_\bullet (F (\phi), F (\psi)) = G_\bullet (F' (\phi), F' (\psi)) = F' (\bulletord \phi \psi)\]
  \end{step}
  証明は以上である.

  命題1.1.9において,$Z = \{\true, \false\}$,$F_0 = V_0$とし,関数$G_\lnot \colon \{\true, \false\} \to \{\true, \false\}$と$G_\bullet \colon \{\true, \false\} \times \{\true, \false\} \to \{\true, \false\}$を
  \[
    G_\lnot \colon
    \begin{aligned}
      \true & {}\mapsto \false \\
      \false & {}\mapsto \true
    \end{aligned} \quad
    G_\lor \colon
    \begin{aligned}
      (\true, \true) & {}\mapsto \true \\
      (\true, \false) & {}\mapsto \true \\
      (\false, \true) & {}\mapsto \true \\
      (\false, \false) & {}\mapsto \false
    \end{aligned} \quad
    G_\land \colon
    \begin{aligned}
      (\true, \true) & {}\mapsto \true \\
      (\true, \false) & {}\mapsto \false \\
      (\false, \true) & {}\mapsto \false \\
      (\false, \false) & {}\mapsto \false
    \end{aligned} \quad
    G_\ra \colon
    \begin{aligned}
      (\true, \true) & {}\mapsto \true \\
      (\true, \false) & {}\mapsto \false \\
      (\false, \true) & {}\mapsto \true \\
      (\false, \false) & {}\mapsto \true
    \end{aligned} \quad
    G_\lra \colon
    \begin{aligned}
      (\true, \true) & {}\mapsto \true \\
      (\true, \false) & {}\mapsto \false \\
      (\false, \true) & {}\mapsto \false \\
      (\false, \false) & {}\mapsto \true
    \end{aligned}
  \]
  と定めれば,定理1.1.7を得る.
\end{excfield}



\begin{excfield}{演習1.1.13}
  \begin{myenum}[n]
    \item $\lnot ((\lnot p_1) \lor p_2)$
    \item 仮にこれが$L$--文であるとする.命題1.1.5より,$\landord p_1 p_2 \lnot p_3 = \landord \phi \psi$なる$L$--文$\phi, \psi$が一意に存在する.よって,$p_1$は$L$--文であるから,$p_2 \lnot p_3$は$L$--文である.しかし,$p_2 \lnot p_3$は命題1.1.5のいずれの場合も満たさないので,$L$--文ではない.矛盾.ゆえに$\landord p_1 p_2 \lnot p_3$は$L$--文ではない.
    \item $(p_1 \land p_2) \ra (((\lnot p_3) \lor p_8) \lra p_3)$
  \end{myenum}
\end{excfield}
